A log format is defined as a line of text where its data is separated by a single char or a string. The log format can also have a specific start and stop sequence to indicate start and end of text, this is especially good when using it over the network.

\hyperlink{class_log__viewer_1_1_log__format}{Log\-\_\-format} is the interface for all inherited formats, override the pure virtual functions to implement a new format\-: 
\begin{DoxyCode}
\textcolor{keyword}{public}:
    \textcolor{keyword}{virtual} QString get\_description() \textcolor{keyword}{const} = 0;
    \textcolor{keyword}{virtual} QString get\_start\_seq() \textcolor{keyword}{const} = 0;
    \textcolor{keyword}{virtual} QString get\_stop\_seq() \textcolor{keyword}{const} = 0;
    \textcolor{keyword}{virtual} QString get\_origin() \textcolor{keyword}{const} = 0;
    \textcolor{keyword}{virtual} \textcolor{keywordtype}{bool} match(\textcolor{keyword}{const} QString first\_line) \textcolor{keyword}{const} = 0;
\textcolor{keyword}{private}:
    \textcolor{keyword}{virtual} QSharedPointer<Log\_item> create\_log\_item(\textcolor{keyword}{const} QString value, \textcolor{keyword}{const} QString origin) \textcolor{keyword}{const} = 0;
\end{DoxyCode}


As seen above a \hyperlink{class_log__viewer_1_1_log__format}{Log\-\_\-format} class decendant is responsible for the creation of its log items.

\hyperlink{class_log__viewer_1_1_log__item}{Log\-\_\-item} serves as the base class for log items. A log item is used to break down the line of text to its predefined data structure. The data structure will eventually end up as columns in the G\-U\-I. This is a list of the data structure, all access to these private variables is done with a getter. 
\begin{DoxyCode}
\textcolor{keywordtype}{int} m\_index;
Log\_type m\_type;
QString m\_text;
QString m\_date;
QString m\_time;
QString m\_file;
QString m\_line;
QString m\_application;
QString m\_module;
QString m\_thread;
QString m\_origin;
\end{DoxyCode}


In the name of simplicity for the end user all formats shall try to auto-\/sense its format when requested. To ensure a valid match regular expressions are used. 
\begin{DoxyCode}
\textcolor{keyword}{virtual} \textcolor{keywordtype}{bool} match(\textcolor{keyword}{const} QString first\_line) \textcolor{keyword}{const} = 0;
\end{DoxyCode}
 As an improvemnet the formats could be loaded on demand via Q\-T plugin architecture.

To add a format you need to modify the \hyperlink{class_log__viewer_1_1_log__format__factory}{Log\-\_\-format\-\_\-factor} class. 
\begin{DoxyCode}
QSharedPointer<Log\_format> Log\_format\_factory::create(\textcolor{keyword}{const} QString first\_line, \textcolor{keyword}{const} QString origin)
\{
    QSharedPointer<Log\_format> format;

    format = QSharedPointer<G3\_log\_format>(\textcolor{keyword}{new} G3\_log\_format(origin));
    \textcolor{keywordflow}{if}(format->match(first\_line))
        \textcolor{keywordflow}{return} format;

    format = QSharedPointer<NRG\_log\_format>(\textcolor{keyword}{new} NRG\_log\_format(origin));
    \textcolor{keywordflow}{if}(format->match(first\_line))
        \textcolor{keywordflow}{return} format;

    format = QSharedPointer<NRG\_legacy\_log\_format>(\textcolor{keyword}{new} NRG\_legacy\_log\_format(origin));
    \textcolor{keywordflow}{if}(format->match(first\_line))
        \textcolor{keywordflow}{return} format;

    \textcolor{comment}{// Add new format here...}

    \textcolor{keywordflow}{return} QSharedPointer<Unknown\_log\_format>(\textcolor{keyword}{new} Unknown\_log\_format());
\}
\end{DoxyCode}
 